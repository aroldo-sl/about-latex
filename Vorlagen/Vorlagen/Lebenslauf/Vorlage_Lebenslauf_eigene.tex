%%Vorlage "Lebenslauf-eigener", v1.03

\documentclass[a4paper,11pt]{scrartcl}
\usepackage[T1]{fontenc}
\usepackage[utf8]{inputenc}
\usepackage[ngerman]{babel}
\usepackage{textcomp}
\usepackage{tabularx,xcolor}
\usepackage{graphicx,setspace}
\usepackage[osf]{libertine}
\usepackage{microtype}
\begin{document}

\definecolor{Maroon}{cmyk}{0, 0.87, 0.68, 0.32}
\onehalfspace

\parindent0mm
\renewcommand{\arraystretch}{1.2}

\begin{figure}
\begin{minipage}{0.6\linewidth}
{\Huge \textsc{\textcolor{Maroon}{Curriculum Vitae}}}\\[0.7cm]
{\Large \textsc{Jason Billings}}\\[0.2cm]
Kesselhof 4\\
D-01234 Stadthausen\\
+49-(0)16\,16\,65\,87\,11\\
jason.billings@woodme.de\\[-1.5cm]
\end{minipage}
\begin{minipage}{0.4\linewidth}
\hspace*{2.4cm} \includegraphics[width=0.6\textwidth]{PortraitLebenslauf.jpg}
\end{minipage}
\end{figure}

\vspace*{0.3cm}

\begin{tabularx}{\textwidth}{p{2.8cm}X}
Geboren am: & 16. Oktober 1966, Stadthausen, Deutschland\\
\end{tabularx}

\vspace*{0.9cm}

\textsc{Ausbildung und wissenschaftlicher Werdegang}\par
\noindent\rule[1ex]{\textwidth}{0.2pt}

\begin{tabularx}{\textwidth}{p{2.8cm}X}

seit Juli 1980 & \textbf{Promotion}, Universität der freien Künste, Thema: \glqq Untersuchung von geeigneten \LaTeX-Vorlagen\grqq\\
Mai 1979 & \textbf{Diplomabschluss} als Diplom-TeXologe, Mittelsoost-Universität; Note: \textbf{2,1} (gut)\\
1979/1980 & \textbf{Diplomarbeit}: \glqq Eingehende Untersuchungen an Tephra aus Anden-Vulkanen\grqq; Betreuer: Prof.~Dr.I. Vulkaan (Institut Vulkanologie, Seismohausen), Prof.~Dr.~L. Schockwelle (Institut für Scherwellen, Epihausen); Note: \textbf{1,3} (sehr gut)\\
2010	& \textbf{Mündliche Prüfungen}\\
	& \quad Vulkanologie: \textbf{2,3} (gut)\\[-0.15cm]
	& \quad Seismologie: \textbf{1,7} (gut)\\[-0.15cm]
	& \quad Meeresgeologie: \textbf{4,0} (ausreichend)\\[-0.15cm]
	& \quad Tiefseetauchen: \textbf{3,6} (ausreichend)\\
12/1976	& \textbf{Vordiplom}\\
04/1973 & \textbf{Beginn Diplom-Studiengang Vulkanologie/Muscheltauchen} an der Mittelsoost-Universität, Habichtshausen\\
\end{tabularx}

\begin{tabularx}{\textwidth}{p{2.8cm}X}
01/1971--10/1971 & \textbf{Grundwehrdienst} 12. Panzerbrigade (Schleswig-Holstein)\\
06/1965--07/1970 & \textbf{Gymnasium} Albrecht-Hanters-Gymnasium, Ockelstedt\\
	& Abschluss: Abitur\\
1960--1964 & \textbf{Grundschule} Pechtermann-Schule, Wieselingen\\
\end{tabularx}
\vspace*{0.5cm}

\textsc{Auslandserfahrung}\par
\noindent\rule[1ex]{\textwidth}{0.2pt}

\begin{tabularx}{\textwidth}{p{2.8cm}X}
07--09/1979 & \textbf{Diplomkartierung: Kartierung der Nord-Mongolei}\\

	& Proterozoische Sedimente der Ufti-Tufti-Formation, Gneise und Gerölle des Jusselbluk-Members.\\
	
02--03/1972 & \textbf{Mitfahrt auf Forschungsschiff \glqq Seestern\grqq{} im Südatlantik}\\

	& Untersuchungen an der Zusammensetzung der Wirbellosen-Faunen unter 4.000\,m.\\
	
\end{tabularx}
\vspace*{0.5cm}

\textsc{Zusätzliche Erfahrung}\par
\noindent\rule[1ex]{\textwidth}{0.2pt}

\begin{tabularx}{\textwidth}{p{2.8cm}X}
1979--1980 & \textbf{Aushilfe am Museum} am Institut für Vulkanologie\\
	& Aussortierung entglaster Obsidiane\\
	
1978 & \textbf{Gesteinsbestimmungspraktikum} am Institut für Vulkanologie\\
	& Unterscheidung eines Diamanten von Torf.\\
\end{tabularx}
	
\vspace*{0.5cm}

\textsc{Wissenschaftliche Veröffentlichungen}\par
\noindent\rule[1ex]{\textwidth}{0.2pt}

\begin{tabularx}{\textwidth}{p{2.8cm}X}

1983 & \textsc{Billings,~J.} \& \textsc{Dean, V.} (1983): Nachweis der ersten auf dem Rücken treibenden toten Pottwale in der Nordsee. -- Australian Journal for Scary Science, 62/1: 443--472.\\

\end{tabularx}
\vspace*{0.5cm}

\textsc{Öffentliche Vorträge}\par
\noindent\rule[1ex]{\textwidth}{0.2pt}

\begin{tabularx}{\textwidth}{p{2.8cm}X}
	02/1980 & \glqq Die Gahnite der West-Mongolei\grqq\ (Mongolischer Mineralogischer Verein) \\

\end{tabularx}
\vspace*{0.5cm}

\newpage
\textsc{Teilnahme an Tagungen und Workshops}\par
\noindent\rule[1ex]{\textwidth}{0.2pt}

\begin{tabularx}{\textwidth}{p{2.8cm}X}
	04/1082 & \textbf{Dead Whales Workshop, Northsea} \\ 
		& Distribution and taphonomic situation of fossil whales on the water surface; Leitung: Dr. Eisbär (Nordpol)\\

\end{tabularx}
\vspace*{0.5cm}

\textsc{Sprachkenntnisse}\par
\noindent\rule[1ex]{\textwidth}{0.2pt}

\begin{tabularx}{\textwidth}{p{2.8cm}X}
	Deutsch & Muttersprache\\
	Englisch & flüssig\\
	Klingonisch & Grundlagen\\
\end{tabularx}
\vspace*{0.5cm}

\textsc{EDV-Kenntnisse}\par
\noindent\rule[1ex]{\textwidth}{0.2pt}

\begin{tabularx}{\textwidth}{p{2.8cm}X}
Betriebssysteme	& Linux\\
Büro-Software & LibreOffice, Gnumeric, \LaTeX\\
Grafik-Software & Inkscape, GIMP\\
Datenbanken, GIS & Specify, Quantum GIS, Viking\\
\end{tabularx}
\vspace*{0.5cm}

\textsc{Führerschein}\par
\noindent\rule[1ex]{\textwidth}{0.2pt}

\begin{tabularx}{\textwidth}{p{2.8cm}X}
Klasse B & eigener PKW
\end{tabularx}
\vspace*{0.5cm}

\textsc{Interessen und Hobbys}\par
\noindent\rule[1ex]{\textwidth}{0.2pt}

%\begin{tabularx}{\textwidth}{p{2.8cm}X}
Walforschung, Vulkanite\\
Schwimmen, auf Hochzeitsfeiern gehen\\
\TeX-Programmierung\\
\dots\ sowie Fischen, Nachdenken und alles was sonst hiermit nichts zu tun hat.
%\end{tabularx}
\vspace*{1cm}

Oderstedt, \today

\end{document}
