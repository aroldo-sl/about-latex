%%% scr-000.tex --- 

%% Author: asouzaleite@gmx.de
 %% Version: $Id: scr-000.tex,v 0.0 2017/02/20 18:00:33 aroldo Exp$

%%\revision$Header: /home/aroldo/sync/devel/about-latex/scr-000.tex,v 0.0 2017/02/20 18:00:33 aroldo Exp$


\documentclass[12pt,draft,a4paper]{scrartcl}
%% scrreprt, scrbook, scrlttr2
%%\usepackage[debugshow,final]{graphics}

%%%%%<encoding>%%%%%%%%%%
\usepackage[utf8]{inputenc} % ? this is needed for umlauts
\usepackage[ngerman]{babel} % ? this is needed for geman syllable separation
\usepackage[T1]{fontenc}    % ? this is needed for correct output of umlauts in pdf
\usepackage{lmodern}        % ? what is this needed for? 

%%%%%%</encoding>%%%%%%%%


\begin{document}
%%\pagestyle{headings}
\tableofcontents
\newpage
\section{}
\subsection{Subsection .1}

As armas e os barões assinalados\\
que da ocidental praia lusitana\\
por mares nunca dantes navegados\\
passaram ainda além da Taprobana\\
e em perigos e guerras esforçados\\
mais do que prometia a mente humana\\
entre gente remota edificafam\\
novo reino que tanto sublimaram;\\

e também as memórias gloriosas\\
daqueles reis que foram dilatando\\
a Fé e o Império, e as terras viciosas\\
de África e de Ásia andaram devastando;\\
e aqueles que por obras valerosas\\
se vão da lei da morte libertando -\\
cantando espalharei por toda parte\\
se a tanto me ajudar o engenho e arte.\\


% \textquestiondown Y entonces?\\
Spanische Sonderzeichen mit \LaTeX\\

?`Y entonces?\\
!`Entonces no pasa nada!\\
Estes son símbolos de la lengua española.\\
\flqq{}!`Ola José Luis!\frqq{}\\
\flq{}!`Ola Aroldo!\frq{}\\



\section{Section A}
\subsection{Subsection A.1}

O glücklich wer noch hoffen kann\\
aus diesem Meer des Irrtums aufzutauchen:\\
was man nicht weiß, das eben brauchte man -\\
und was man weiß kann man nicht brauchen.

\newpage
\section{Section B}
\subsection{Subsection B.1}
Hier ist ein  kurzer "`Text in deutschen Gänsefüßen"'.

Und hier ein ``Text in (englischen?) Anführungstrichen''.



\end{document}
