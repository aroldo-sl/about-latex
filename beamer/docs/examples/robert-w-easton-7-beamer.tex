\documentclass{beamer}

\usetheme{Berlin}

\usepackage{hyperref}

\title[$\sqrt{2}$ is Irrational]{$\sqrt{2}$ is Irrational\\{\small A Sample \texttt{BEAMER} Presentation}}
\author{Math 351, Winter 2013}
\date{February 26, 2013}

%%%%%%%%%%%%%%%%%%%%%%%%%%%%%%%%%%%%%%%%%%%%%%

\begin{document}

\frame{
	\titlepage
}

%%%%%%%%%%%%%%%%%%%%%%%%%%%%%%%%%%%%%%%%%%%%%%
%%%%%%%%%%%%%%%%%%%%%%%%%%%%%%%%%%%%%%%%%%%%%%

% Here we have separated frames by a row of %s and sections by two rows of %s, but you can certainly choose your own internal formatting!

\section*{Outline}

\frame{
	\tableofcontents[pausesections]
	
% The argument is causing the table of contents to appear one section at a time, with pauses between the sections.
}

%%%%%%%%%%%%%%%%%%%%%%%%%%%%%%%%%%%%%%%%%%%%%%
%%%%%%%%%%%%%%%%%%%%%%%%%%%%%%%%%%%%%%%%%%%%%%

\section{Introduction}

\frame{
	\frametitle{Preview}
	In this talk, we will prove one of the great theorems in all of mathematics.  \\[.5in]
	\pause
	Along the way we will collect some necessary definitions.
}

%%%%%%%%%%%%%%%%%%%%%%%%%%%%%%%%%%%%%%%%%%%%%%

\frame{
	\frametitle{Some history}
	\begin{itemize}[<+->]
	\item The main result of this talk has been known for at least two thousand years.  
	\item It profoundly changed the Greek view of the universe.
	\item Legend has it that Hippasus of Metapontum was drowned for his belief in it.\footnote<.->{Morris Kline, {\itshape Mathematical Thought from Ancient to Modern Times}, Oxford University Press, 1990, pg. 32}
	\end{itemize}
	
% Notice how this itemize is working. The argument <+-> is indicating that each item should first appear on the slide given by its internal item number (that's the +) and should remain shown until the end of the frame (that's the -). In general, the argument <m-n> instructs LaTeX to begin showing the indicated output on slides m through n of the given frame.
}

%%%%%%%%%%%%%%%%%%%%%%%%%%%%%%%%%%%%%%%%%%%%%%
%%%%%%%%%%%%%%%%%%%%%%%%%%%%%%%%%%%%%%%%%%%%%%

\section{Rational Numbers}

\frame{
	\frametitle{Definition}
	\begin{itemize}
	\item<1-> A \alert{rational number} $x$ is any number that can be represented as the ratio of two integers.
	\item<2-> For example $x = \frac{22}{7}$ and $x = \frac{2009}{2010}$ are rational numbers.  
	\item<3-> By contrast $x = \pi$ is \alert{irrational}.  \onslide<4>(See Wikipedia for a collection of various proofs.)
	\end{itemize}
	
% Here, we are manually indicating when each item should appear.  For example, item 2 is set to appear on slide 2 and remain displayed until the end of the frame.  Notice how we made the last remark only appear on slide 4 of this frame.
	
}

%%%%%%%%%%%%%%%%%%%%%%%%%%%%%%%%%%%%%%%%%%%%%%
%%%%%%%%%%%%%%%%%%%%%%%%%%%%%%%%%%%%%%%%%%%%%%

\section{Relatively Prime Numbers}

\frame{
	\frametitle{Definition}
	Two integers $a$ and $b$ are said to be \alert{relatively prime} if they do not share any common factors other than $\pm 1$. \\[.5in]
	\pause
	For example, $a = 17$ and $b = 256$ are relatively prime, whereas $a = 15$ and $a = 24$ are not (as $3$ divides both).
}

%%%%%%%%%%%%%%%%%%%%%%%%%%%%%%%%%%%%%%%%%%%%%%
%%%%%%%%%%%%%%%%%%%%%%%%%%%%%%%%%%%%%%%%%%%%%%

\section{Main Result}

\frame{
	\frametitle{Infinite descent}
	\begin{theorem}
		$\sqrt2$ is irrational.
	\end{theorem}

	\pause
	\begin{proof}
		Suppose, by way of contradiction, that $\sqrt2$ is rational.\\
		
		\pause Then there exist relatively prime integers $a, b$ with $\sqrt{2} = \frac{a}{b}$.
		
		\vspace{-0.1in}
		\[
			\begin{array}{cl}
			\onslide<3->
			\implies & 2b^2 = a^2 \\
			\onslide<4->
			 \implies & 2 \mid a \\
			 \onslide<5->
			 \implies & a = 2m \quad \text{for some integer }m\\
			 \onslide<6->
			 \implies & b^2 = 2m^2 \\
			\onslide<7->
			 \implies & 2 \mid b
			\end{array}
		\]
		\onslide<8>But we assumed $a, b$ were relatively prime! Contradiction!
	\end{proof}
}

%%%%%%%%%%%%%%%%%%%%%%%%%%%%%%%%%%%%%%%%%%%%%%
%%%%%%%%%%%%%%%%%%%%%%%%%%%%%%%%%%%%%%%%%%%%%%

\section{References}

\frame{
\frametitle{\texttt{BEAMER} References}
\begin{itemize}[<+->]
\item Official documentation for the \texttt{BEAMER} class\\Available on PolyLearn and on the web
\item Wikibooks,``LaTeX/Presentations" \url{http://en.wikibooks.org/wiki/LaTeX/Presentations}
\item Rouben Rostamian,``A \texttt{BEAMER} Quickstart" \url{http://www.math.umbc.edu/~rouben/beamer/}
\item Charles Batts,``A Beamer Tutorial in Beamer" \url{http://www.uncg.edu/cmp/reu/presentations/Charles\%20Batts\%20-\%20Beamer\%20Tutorial.pdf}
\end{itemize}
}

%%%%%%%%%%%%%%%%%%%%%%%%%%%%%%%%%%%%%%%%%%%%%%

\frame{
\frametitle{Thanks!}
A heartfelt thanks to:\pause
\begin{itemize}[<+->]
\item Professor Sherman. This sample \texttt{BEAMER} presentation is a modified version of a file he created for the Winter 2010 section of Math 351.
\item The Greeks, for not drowning me.
\end{itemize}
}

\end{document}

